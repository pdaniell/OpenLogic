% Part:sets-functions-relations
% Chapter: sets
% Section: non-enumerability

\documentclass[../../../include/open-logic-section]{subfiles}

\begin{document}

\olfileid{sfr}{siz}{nen}

\olsection{\printtoken{S}{nonenumerable} Sets}

Some sets, such as the set $\Int^+$ of positive integers, are
infinite. So far we've seen examples of infinite sets which were all
!!{enumerable}. However, there are also infinite sets which do not
have this property. Such sets are called \emph{!!{nonenumerable}}.

First of all, it is perhaps already surprising that there are
!!{nonenumerable} sets.  For any !!{enumerable} set~$X$ there is
!!a{surjective} function $f \colon \Int^+ \to X$.  If a set is
!!{nonenumerable} there is no such function.  That is, no function
mapping the infinitely many !!{element}s of~$\Int^+$ to~$X$ can
exhaust all of~$X$.  So there are ``more'' !!{element}s of~$X$ than
the infinitely many positive integers.

How would one prove that a set is !!{nonenumerable}? You have to show
that no such surjective function can exist. Equivalently, you have to
show that the elements of~$X$ cannot be enumerated in a one way
infinite list.  The best way to do this is to show that every list of
!!{element}s of~$X$ must leave at least one element out; or that no
function $f\colon \Int^+ \to X$ can be surjective.  We can do this
using Cantor's \emph{diagonal method}.  Given a list of !!{element}s
of $X$, say, $x_1$, $x_2$, \dots, we construct another element of~$X$
which, by its construction, cannot possibly be on that list.

Our first example is the set~$\Bin^\omega$ of all infinite, non-gappy
sequences of $0$'s and $1$'s.

\begin{thm}
\ollabel{thm-nonenum-bin-omega}
$\Bin^\omega$~is !!{nonenumerable}.
\end{thm}

\begin{proof}
Suppose, by way of contradiction, that $\Bin^\omega$ is
!!{enumerable}, i.e., suppose that there is a list $s_{1}$, $s_{2}$,
$s_{3}$, $s_{4}$, \dots{} of all !!{element}s of~$\Bin^\omega$.  Each
of these $s_i$ is itself an infinite sequence of $0$'s and~$1$'s.
Let's call the $j$-th element of the $i$-th sequence in this list
$s_i(j)$. Then the $i$-th sequence~$s_i$ is
\[
s_i(1), s_i(2), s_i(3), \dots
\]

We may arrange this list, and the elements of each sequence $s_i$ in
it, in an array:
\[
\begin{array}{c|c|c|c|c|c}
& 1 & 2 & 3 & 4 & \dots \\\hline
1 & \mathbf{s_{1}(1)} & s_{1}(2) & s_{1}(3) & s_1(4) & \dots \\\hline
2 & s_{2}(1)& \mathbf{s_{2}(2)} & s_2(3) & s_2(4) & \dots \\\hline
3 & s_{3}(1)& s_{3}(2) & \mathbf{s_3(3)} & s_3(4) & \dots \\\hline
4 & s_{4}(1)& s_{4}(2) & s_4(3) & \mathbf{s_4(4)} & \dots \\\hline
\vdots & \vdots & \vdots & \vdots & \vdots & \mathbf{\ddots}
\end{array}
\]
The labels down the side give the number of the sequence in the list
$s_1$, $s_2$, \dots; the numbers across the top label the !!{element}s
of the individual sequences. For instance, $s_{1}(1)$ is a name for
whatever number, a $0$ or a~$1$, is the first !!{element} in the
sequence $s_{1}$, and so on.

Now we construct an infinite sequence, $\overline{s}$, of $0$'s and
$1$'s which cannot possibly be on this list.  The definition of
$\overline{s}$ will depend on the list $s_1$, $s_2$, \dots.  Any
infinite list of infinite sequences of $0$'s and $1$'s gives rise to
an infinite sequence~$\overline{s}$ which is guaranteed to not appear
on the list.

To define $\overline{s}$, we specify what all its !!{element}s are,
i.e., we specify $\overline{s}(n)$ for all $n \in \Int^+$.  We do this
by reading down the diagonal of the array above (hence the name
``diagonal method'') and then changing every $1$ to a $0$ and every
$1$ to a~$0$. More abstractly, we define $\overline{s}(n)$ to be $0$
or $1$ according to whether the $n$-th !!{element} of the diagonal,
$s_n(n)$, is $1$ or $0$.
\[
\overline{s}(n) =
\begin{cases}
1 & \text{if $s_{n}(n) = 0$}\\
0 & \text{if $s_{n}(n) = 1$}.
\end{cases}
\]
If you like formulas better than definitions by cases, you could also
define $\overline{s}(n) = 1 - s_n(n)$.

Clearly $\overline{s}$ is a non-gappy infinite sequence of $0$'s and
$1$'s, since it is just the mirror sequence to the sequence of $0$'s
and $1$'s that appear on the diagonal of our array.  So $\overline{s}$
is !!a{element} of~$\Bin^\omega$.  But it cannot be on the list $s_1$,
$s_2$, \dots{} Why not?

It can't be the first sequence in the list, $s_1$, because it differs from
$s_1$ in the first !!{element}.  Whatever $s_1(1)$ is, we defined
$\overline{s}(1)$ to be the opposite.  It can't be the second
sequence in the list, because $\overline{s}$ differs from $s_2$ in the second
element: if $s_2(2)$ is $0$, $\overline{s}(2)$ is $1$, and vice
versa. And so on.

More precisely: if $\overline{s}$ were on the list, there would be
some $k$ so that $\overline{s} = s_{k}$.  Two sequences are identical
iff they agree at every place, i.e., for any~$n$, $\overline{s}(n) =
s_{k}(n)$.  So in particular, taking $n = k$ as a special case,
$\overline{s}(k) = s_{k}(k)$ would have to hold. $s_k(k)$ is either
$0$ or~$1$. If it is $0$ then $\overline{s}(k)$ must be~$1$---that's
how we defined $\overline{s}$. But if $s_k(k) = 1$ then, again because
of the way we defined $\overline{s}$, $\overline{s}(k) = 0$. In either
case $\overline{s}(k) \neq s_{k}(k)$.

We started by assuming that there is a list of !!{element}s of
$\Bin^\omega$, $s_1$, $s_2$, \dots{} From this list we constructed a
sequence~$\overline{s}$ which we proved cannot be on the list.  But it
definitely is a sequence of $0$'s and $1$'s if all the $s_i$ are
sequences of $0$'s and $1$'s, i.e., $\overline{s} \in
\Bin^\omega$. This shows in particular that there can be no list of
\emph{all} !!{element}s of~$\Bin^\omega$, since for any such list we
could also construct a sequence~$\overline{s}$ guaranteed to not be on
the list, so the assumption that there is a list of all sequences
in~$\Bin^\omega$ leads to a contradiction.
\end{proof}

\begin{explain}
This proof method is called ``diagonalization'' because it uses the
diagonal of the array to define~$\overline{s}$. Diagonalization need
not involve the presence of an array: we can show that sets are not
!!{enumerable} by using a similar idea even when no array and no
actual diagonal is involved.
\end{explain}

\begin{thm}
\ollabel{thm-nonenum-pownat}
$\Pow{\Int^+}$ is not !!{enumerable}.
\end{thm}

\begin{proof}
We proceed in the same way, by showing that for every list of subsets
of~$\Int^+$ there is a subset of $\Int^+$ which cannot be on the list.
Suppose the following is a given list of subsets of~$\Int^+$:
\[
Z_{1}, Z_{2}, Z_{3}, \dots
\]
We now define a set $\overline{Z}$ such that for any $n \in \Int^+$,
$n \in \overline{Z}$ iff $n \notin Z_{n}$:
\[
\overline{Z} = \Setabs{n \in \Int^+}{n \notin Z_n}
\]
$\overline{Z}$ is clearly a set of positive integers, since by
assumption each~$Z_n$ is, and thus $\overline{Z} \in
\Pow{\Int^+}$. But $\overline{Z}$ cannot be on the list.  To show
this, we'll establish that for each $k \in \Int^+$, $\overline{Z} \neq
Z_k$. 

So let $k \in \Int^+$ be arbitrary. We've defined $\overline{Z}$ so
that for any $n \in \Int^+$, $n \in \overline{Z}$ iff $n \notin Z_n$.
In particular, taking $n=k$, $k \in \overline{Z}$ iff $k \notin Z_k$.
But this shows that $\overline{Z} \neq Z_k$, since $k$ is !!a{element}
of one but not the other, and so $\overline{Z}$ and $Z_k$ have
different !!{element}s. Since $k$ was arbitrary, $\overline{Z}$ is not
on the list $Z_1$, $Z_2$, \dots
\end{proof}

\begin{explain}
The preceding proof did not mention a diagonal, but you can think of
it as involving a diagonal if you picture it this way: Imagine the
sets $Z_1$, $Z_2$, \dots, written in an array, where each
!!{element}~$j \in Z_i$ is listed in the~$j$-th column. Say the first
four sets on that list are $\{1,2,3,\dots\}$, $\{2, 4, 6, \dots\}$,
$\{1,2,5\}$, and $\{3,4,5,\dots\}$. Then the array would begin with
\[
\begin{array}{r@{}rrrrrrr}
  Z_1 = \{ & \mathbf{1}, & 2, & 3, & 4, & 5, & 6, & \dots\}\\
  Z_2 = \{ &  & \mathbf{2}, &  & 4, &  & 6, & \dots\}\\
  Z_3 = \{ & 1, & 2, &  &  & 5\phantom{,} &  & \}\\
  Z_4 = \{ &  &  & 3, & \mathbf{4}, & 5, & 6, & \dots\}\\
  \vdots & & & & & \ddots
\end{array}
\]
Then $\overline{Z}$ is the set obtained by going down the diagonal,
leaving out any numbers that appear along the diagonal and include
those $j$ where the array has a gap in the $j$-th row/column. In the
above case, we would leave out $1$ and $2$, include~$3$, leave
out~$4$, etc.
\end{explain}

\begin{prob}
Show that $\Pow{\Nat}$ is !!{nonenumerable} by a diagonal argument.
\end{prob}

\begin{prob}
Show that the set of functions $f \colon \Int^+ \to \Int^+$ is
!!{nonenumerable} by an explicit diagonal argument. That is, show that
if $f_1$, $f_2$, \dots, is a list of functions and each $f_i\colon
\Int^+ \to \Int^+$, then there is some $\overline{f}\colon \Int^+ \to
\Int^+$ not on this list.
\end{prob}

\end{document}
