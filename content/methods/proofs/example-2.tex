% Part: methods
% Chapter: proofs
% Section: example-2

\documentclass[../../../include/open-logic-section]{subfiles}

\begin{document}

\olfileid{mth}{prf}{ex2}

\olsection{Another Example} 

\begin{prop}
If $X \subseteq Z$, then $X \cup (Z \setminus X) = Z$.
\end{prop}  

\begin{proof}
We begin by observing that this is a conditional statement. It is
tacitly universally quantified: the proposition holds for all sets $X$
and $Z$. So $X$ and $Z$ are variables for arbitrary sets. To prove
such a statement, we assume the antecedent and prove the consequent.

\begin{quote}  
Suppose that $X \subseteq Z$. We want to show that $X \cup (Z \setminus
X) = Z$.
\end{quote}

What do we know? We know that $X \subseteq Z$. Let's unpack the
definition of~$\subseteq$: the assumption means that all !!{element}s
of~$X$ are also elements of~$Z$. Let's write this down---it's an
important fact that we'll use throughout the proof.

\begin{quote}
By the definition of~$\subseteq$, since $X \subseteq Z$, for all $z$,
if $z \in X$, then $z \in Z$.
\end{quote}

We've unpacked all the definitions that are given to us in the
assumption. Now we can move onto the conclusion. We want to show that
$X \cup (Z \setminus X) = Z$, and so we set up a proof similarly to
the last example: we show that every !!{element} of $X \cup (Z
\setminus X)$ is also !!a{element} of~$Z$ and, conversely, every
!!{element} of $Z$ is !!a{element} of $X \cup (Z \setminus X)$. We can
shorten this to: $X \cup (Z \setminus X) \subseteq Z$ and $Z \subseteq
X \cup (Z \setminus X)$. (Here were doing the opposite of unpacking a
definition, but it makes the proof a bit easier to read.)  Since this
is a conjunction, we have to prove both parts. To show the first part,
i.e., that every !!{element} of $X \cup (Z \setminus X)$ is also
!!a{element} of~$Z$, we assume that for an arbitrary $z$ that $z \in X
\cup (Z \setminus X)$ and show that $z \in Z$. By the definition of
$\cup$, we can conclude that $z \in X$ or $z \in Z \setminus X$ from
$z \in X \cup (Z \setminus X)$. You should now be getting the hang of
this.

\begin{quote}
$X \cup (Z \setminus X) = Z$ iff $X \cup (Z \setminus X) \subseteq Z$
  and $Z \subseteq (X \cup (Z \setminus X)$.  First we prove that $X
  \cup (Z \setminus X) \subseteq Z$.  Let $z \in X \cup (Z \setminus
  X)$. So, either $z \in X$ or $z \in (Z \setminus X)$.
\end{quote}

We've arrived at a disjunction, and from it we want to prove that $z
\in Z$. We do this using proof by cases.

\begin{quote}
Case 1: $z \in X$. Since for all $z$, if $z \in X$, $z
\in Z$, we have that $z \in Z$.
\end{quote}

Here we've used the fact recorded earlier which followed from the
hypothesis of the proposition that $X \subseteq Z$.  The first case is
complete, and we turn to the second case, $z \in (Z \setminus X)$.
Recall that $Z \setminus X$ denotes the \emph{difference} of the two
sets, i.e., the set of all !!{element}s of $Z$ which are not
!!{element}s of $X$.  Let's use state what the definition gives
us. But an element of $Z$ not in~$X$ is in particular !!a{element}
of~$Z$.

\begin{quote}
Case 2: $z \in (Z \setminus X)$.  This means that $z \in Z$ and $z
\notin X$. So, in particular, $z \in Z$.
\end{quote}

Great, we've solved the first direction. Now for the second
direction. Here we prove that $Z \subseteq X \cup (Z \setminus X)$.
So we assume that $z \in Z$ and prove that $z \in X \cup (Z \setminus
X)$.

\begin{quote}
Now let $z \in Z$. We want to show that $z \in X$ or $z \in Z
\setminus X$.
\end{quote}

Since all elements of $X$ are also elements of $Z$, and $Z \setminus
X$ is the set of all things that are elements of $Z$ but not $X$, it
follows that $z$ is either in $X$ or $Z \setminus X$.  But this may be
a bit unclear if you don't already know why the result is true.  It
would be better to prove it step-by-step.  It will help to use a
simple fact which we can state without proof: $z \in X$ or $z \notin
X$. This is called the principle of excluded middle: for any
statement~$p$, either $p$ is true or its negation is true. (Here, $p$
is the statement that $z \in X$.)  Since this is a disjunction, we can
again use proof-by-cases.

\begin{quote}
Either $z \in X$ or $z \notin X$. In the former case, $z \in X \cup (Z
\setminus X)$. In the latter case, $z \in Z$ and $z \notin X$, so $z
\in Z \setminus X$.  But then $z \in X \cup (Z \setminus X)$.
\end{quote}

Our proof is complete: we have shown that $X \cup (Z \setminus X) = Z$.
\end{proof}

\end{document}
