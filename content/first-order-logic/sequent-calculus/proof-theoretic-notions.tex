% Part: first-order-logic
% Chapter: sequent-calculus
% Section: proof-theoretic-notions

\documentclass[../../../include/open-logic-section]{subfiles}

\begin{document}

\begin{editorial}
  This section collects the definitions of the provability relation and
  consistency for natural deduction.
\end{editorial}

\olfileid{fol}{seq}{ptn}
\olsection{Proof-Theoretic Notions}

\begin{explain}
Just as we've defined a number of important semantic notions
(validity, entailment, satisfiabilty), we now define corresponding
\emph{proof-theoretic notions}.  These are not defined by appeal to
satisfaction of !!{sentence}s in !!{structure}s, but by appeal to the
!!{derivability} or !!{nonderivability} of certain sequents.  It was
an important discovery, due to G\"odel, that these notions coincide.
That they do is the content of the \emph{completeness theorem}.
\end{explain}

\begin{defn}[Theorems]
!!^a{sentence}~$!A$ is a \emph{theorem} if there is !!a{derivation}
in~$\Log{LK}$ of the sequent $\quad \Sequent !A$.  We write $\Proves
!A$ if $!A$ is a theorem and $\Proves/ !A$ if it is not.
\end{defn}

\begin{defn}[!!^{derivability}]
!!^a{sentence}~$!A$ is \emph{!!{derivable} from} a set of
!!{sentence}s~$\Gamma$, $\Gamma \Proves !A$, iff there is a
finite subset~$\Gamma_0 \subseteq \Gamma$ and a sequence $\Gamma_0'$
of the !!{sentence}s in~$\Gamma_0$ such that $\Log{LK}$ !!{derive}s
$\Gamma_0' \Sequent !A$.  If $!A$ is not !!{derivable} from $\Gamma$
we write $\Gamma \Proves/ !A$.
\end{defn}

Because of the contraction, weakening, and exchange rules, the order
and number of !!{sentence}s in~$\Gamma_0'$ does not matter: if a
sequent $\Gamma_0' \Sequent !A$ is !!{derivable}, then so is
$\Gamma_0'' \Sequent !A$ for any $\Gamma_0''$ that contains the same
!!{sentence}s as~$\Gamma_0'$.  For instance, if $\Gamma_0 = \{!B, !C\}$
then both $\Gamma_0' = \tuple{!B, !B, !C}$ and $\Gamma_0'' =
\tuple{!C, !C, !B}$ are sequences containing just the !!{sentence}s
in~$\Gamma_0$. If a sequent containing one is !!{derivable}, so is the
other, e.g.:
\begin{prooftree}
  \AxiomC{}
  \Deduce$!B, !B, !C \fCenter !A$
  \RightLabel{\LeftR{\Contraction}}
  \UnaryInf$!B, !C \fCenter !A$
  \RightLabel{\LeftR{\Exchange}}
  \UnaryInf$!C, !B \fCenter !A$
  \RightLabel{\LeftR{\Weakening}}
  \UnaryInf$!C, !C, !B \fCenter !A$
\end{prooftree}
From now on we'll say that if $\Gamma_0$ is a finite set of
!!{sentence}s then $\Gamma_0 \Sequent !A$ is any sequent where the
antecedent is a sequence of !!{sentence}s in~$\Gamma_0$ and tacitly include
contractions, exchanges, and weakenings if necessary.

\begin{defn}[Consistency]
A set of sentences~$\Gamma$ is \emph{inconsistent} iff there is a
finite subset~$\Gamma_0 \subseteq \Gamma$ such that $\Log{LK}$
!!{derive}s $\Gamma_0 \Sequent \quad$. If $\Gamma$ is not
inconsistent, i.e., if for every finite $\Gamma_0 \subseteq \Gamma$,
$\Log{LK}$ does not !!{derive} $\Gamma_0 \Sequent \quad$, we say it is
\emph{consistent}.
\end{defn}

\begin{prop}
\ollabel{prop:prov-incons}
$\Gamma \Proves !A$ iff $\Gamma \cup \{\lnot !A\}$ is inconsistent.
\end{prop}

\begin{proof}
First suppose $\Gamma \Proves !A$, i.e., there is
!!a{derivation}~$\pi_0$ of $\Gamma \Sequent !A$. By adding a
$\LeftR{\lnot}$ rule, we obtain !!a{derivation} of~$\lnot !A, \Gamma
\Sequent \quad$, i.e., $\Gamma \cup \{\lnot !A\}$ is inconsistent.

If $\Gamma \cup \{\lnot A\}$ is inconsistent, there is
!!a{derivation}~$\pi_1$ of $\lnot !A, \Gamma \Sequent \quad$. The
following is !!a{derivation} of $\Gamma \Sequent !A$:
\begin{prooftree}
  \Axiom$!A \fCenter !A$
  \RightLabel{\RightR{\lnot}}
  \UnaryInf$\fCenter !A, \lnot !A$
  \AxiomC{}
  \RightLabel{$\pi_1$}
  \Deduce$\lnot !A, \Gamma \fCenter$
  \RightLabel{\Cut}
  \BinaryInf$\Gamma \fCenter !A$
\end{prooftree}
\end{proof}

\begin{prop}
\ollabel{prop:incons}
$\Gamma$ is inconsistent iff $\Gamma \Proves {!A}$ for every
  sentence~$!A$.
\end{prop}

\begin{proof}
Exercise.
\end{proof}

\begin{prob}
Prove \olref[fol][seq][ptn]{prop:incons}
\end{prob}

\begin{prop}[Reflexivity]
\ollabel{prop:reflexivity}
If $!A \in \Gamma$, then $\Gamma \Proves !A$.
\end{prop}

\begin{proof}
The initial sequent $!A \Sequent !A$ is !!{derivable}, and $\{!A\}
\subseteq \Gamma$.
\end{proof}
  
\begin{prop}[Monotony]
\ollabel{prop:monotony}
If $\Gamma \subseteq \Delta$ and $\Gamma \Proves !A$, then $\Delta
\Proves !A$.
\end{prop}

\begin{proof}
Any finite $\Gamma_0 \subseteq \Gamma$ is also a finite subset
of~$\Delta$, so !!a{derivation} of $\Gamma_0 \Sequent !A$ also shows
$\Delta \Proves !A$.
\end{proof}

\begin{prop}[Transitivity]
\ollabel{prop:transitivity}
If $\Gamma \Proves !A$ for every $!A \in \Delta$ and $\Delta \Proves
!B$, then $\Gamma \Proves !B$.
\end{prop}

\begin{proof}
If $\Delta \Proves !B$, then for some finite subset $\Delta_0
\subseteq \Delta$, there is !!a{derivation} of $\Delta_0 \Sequent !B$.
We show that $\Gamma \Proves !B$ by induction on the number~$n$ of
!!{sentence}s in~$\Delta_0$.

If $n=0$, then $\Delta_0$ is empty, and $\Sequent !B$ is
provable. Since $\emptyset \subseteq \Gamma$, $\Gamma \Proves !B$.

Otherwise, pick $!A \in \Delta_0$ and let $\Delta_1 = \Delta_0
\setminus \{!A\}$. There is !!a{derivation}~$\pi_0$ of $!A, \Delta_1
\Sequent !B$. We obtain the !!{derivation}~$\pi_1$:
\begin{prooftree}
  \AxiomC{}
  \RightLabel{$\pi$}
  \Deduce$!A, \Delta_1 \fCenter !B$
  \RightLabel{\RightR{\lif}}
  \UnaryInf$\Delta_1 \fCenter !A \lif !B$
\end{prooftree}
Since the number of !!{sentence}s in $\Delta_1$ is $n-1$, the
inductive hypothesis applies: there is a !!{derivation}~$\pi_2$ of
$\Gamma_0 \Sequent !A \lif !B$ for some $\Gamma_0 \subseteq
\Gamma$. Since $\Gamma \Proves !A$ there is also
!!a{derivation}~$\pi_3$ of $\Gamma_1 \Sequent !A$. Now consider:
\begin{prooftree}
  \AxiomC{}
  \RightLabel{$\pi_2$}
  \Deduce$\Gamma_0 \fCenter !A \lif !B$
  \AxiomC{}
  \RightLabel{$\pi_3$}
  \Deduce$\Gamma_1 \fCenter !A$
  \Axiom$!B \fCenter !B$
  \RightLabel{\LeftR{\lif}}
  \BinaryInf$!A \lif !B, \Gamma_1 \fCenter !B$
  \RightLabel{\Cut}
  \BinaryInf$\Gamma_0, \Gamma_1 \fCenter !B$
\end{prooftree}
This shows $\Gamma \Proves !B$.
\end{proof}

\begin{prop}[Compactness]
  \ollabel{prop:proves-compact}
  \begin{enumerate}
  \item If $\Gamma \Proves !A$ then there is a finite subset $\Gamma_0
    \subseteq \Gamma$ such that $\Gamma_0 \Proves !A$.
  \item If every finite subset of~$\Gamma$ is
    consistent, then $\Gamma$ is consistent.
  \end{enumerate}
\end{prop}

\begin{proof}
  \begin{enumerate}
    \item If $\Gamma \Proves !A$, then there is a finite subset
      $\Gamma_0 \subseteq \Gamma$ such that the sequent $\Gamma_0
      \Sequent !A$ has !!a{derivation}. Consequently, $\Gamma_0
      \Proves !A$.
    \item If $\Gamma$ is inconsistent, there is a finite
      subset~$\Gamma_0 \subseteq \Gamma$ such that $\Log{LK}$
      !!{derive}s $\Gamma_0 \Sequent \quad$. But then $\Gamma_0$ is a
      finite subset of~$\Gamma$ that is inconsistent.
  \end{enumerate}
\end{proof}

\end{document}
