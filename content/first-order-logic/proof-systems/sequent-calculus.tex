% Part: first-order-logic
% Chapter: proof-systems
% Section: seqeunt-calculus

\documentclass[../../../include/open-logic-section]{subfiles}

\begin{document}

\olfileid{fol}{prf}{seq}

\olsection{The Sequent Calculus}

While many !!{derivation} systems operate with arrangements of
!!{sentence}s, the sequent calculus operates with \emph{sequents}. A
sequent is an expression of the form
\[
!A_1, \dots, !A_m \Sequent !B_1, \dots, !B_m,
\]
that is a pair of sequences of !!{sentence}s, separated by the sequent
symbol~$\Sequent$. Either sequence may be empty.  !!^a{derivation} in
the sequent calculus is a tree of sequents, where the topmost sequents
are of a special form (they are called ``initial sequents'' or
``axioms'') and every other sequent follows from the sequents
immediately above it by one of the rules of inference. The rules of
inference either manipulate the !!{sentence}s in the sequents (adding,
removing, or rearranging them on either the left or the right), or
they introduce a complex !!{formula} in the conclusion of the rule.
For instance, the $\LeftR{\land}$ rule allows the inference from $!A,
\Gamma \Sequent \Delta$ to $\!A \land !B, \Gamma \Sequent \Delta$, and
the $\RightR{\lif}$ allows the inference from $!A, \Gamma \Sequent
\Delta, !B$ to $\Gamma \Sequent \Delta, !A \lif !B$, for any $\Gamma$,
$\Delta$, $!A$, and~$!B$. (In particular, $\Gamma$ and~$\Delta$ may be
empty.)

The $\Proves$ relation based on the sequent calculus is defined as
follows: $\Gamma \Proves !A$ iff there is some sequence $\Gamma_0$
such that every $!A$ in $\Gamma_0$ is in~$\Gamma$ and there is a
!!{derivation} with the sequent~$\Gamma_0 \Sequent !A$ at its root.
$!A$ is a theorem in the sequent calculus if the sequent~$\Sequent !A$
has !!a{derivation}. For instance, here is !!a{derivation} that shows
that $\Proves (!A \land !B) \lif !A$:
\begin{prooftree}
  \Axiom$!A \fCenter !A$
  \RightLabel{\LeftR{\land}}
  \UnaryInf$!A \land !B \fCenter !A$
  \RightLabel{\RightR{\lif}}
  \UnaryInf$\fCenter (!A \land !B) \lif !A$
\end{prooftree}

A set $\Gamma$ is inconsistent in the sequent calculus if there is
!!a{derivation} of $\Gamma_0 \Sequent$ (where every $!A \in \Gamma_0$
is in~$\Gamma$ and the right side of the sequent is empty).  Using the
rule \RightR{\Weakening}, any !!{sentence} can be !!{derive}d from an
inconsistent set.

The sequent calculus was invented in the 1930 by Gerhard Gentzen.
Because of its systematic and symmetric design, it is a very useful
formalism for developing a theory of !!{derivation}s. It is relatively
easy to find !!{derivation}s in the sequent calculus, but these
!!{derivation}s are often hard to read and their connection to proofs
are sometimes not easy to see. It has proved to be a very elegant
approach to !!{derivation} systems, however, and many logics have
sequent calculus systems.

\end{document}
